% !Mode:: "TeX:UTF-8"

\chapter{绪论}[Introduction]

\section{课题背景及研究的目的和意义}[Background]
知识图谱作为人工智能的重要基石,近些年得到了不断的发展和应用。特别是搜索引擎开始
利用知识图谱来改进搜索的质量,加快搜索的速度,并且实现语义搜索和推理。

知识图谱是由语义表示的网络,它能够帮助使用者搜索知识网络中的所有知识。知识网络是
由许多个实体,属性和属性值构成的。有全局唯一的标识符URI来表示一个实体,属性是用
来描述实体的内在特性或实体之间的关联关系。W3C提出了一种资源描述框架RDF\cite{rdf}
来表示知识图谱网络,一条知识表示为一个SPO三元组(Subject,Predicate,Object)。
如果从图的角度来看RDF数据的话,RDF的实体或属性值是图的顶点,RDF的属性是图的边。
为了方便RDF数据的检索,W3C组织还提出了一种类似于SQL的形式化查询语言SPARQL\cite{sparql}。
SPARQL查询的基本单位是三元组模式(triple pattern),多个三元组模式通过多种运
算符形成更多复杂的图模式。SPARQL查询是将查询问题转化为图上的子图匹配问题,因为
这一过程可以看作在RDF图中找到全部满足查询图模式的匹配。

随着知识图谱应用的广泛使用,到目前为止,规模为百万顶点(106)和上亿条边(108)的知
识图谱数据集已经比较常见\cite{MassiveData},根据链接开放数据2018年8月发布的LOD云图中很多知识图
谱数据集规模超过10亿条三元组。例如,维基百科知识图谱DBpedia(>30亿条)、地理信息
知识图谱LinkedGeoData(>30亿条)和蛋白质知识图谱UniProt(>130亿条)等。在数据量
如此巨大的情况下,实际应用场景对图数据库的查询性能提出了更高的要求。在此情况下,对
图数据库性能的优化,尤其是对SPARQL的查询优化显得尤为重要。

近年来人工智能的迅速发展为数据库查询问题提供了新的机会\cite{MLforDB}。人工智能技术强大的适应
性和数据表征能力能够智能地对大量、复杂、动态的数据和工作负载进行深入的分析。在现有
工作中,已有许多基于机器学习的数据管理技术被提出。研究人员分别从数据库系统调优和查
询优化两方面出发,提出了如SageDB\cite{SageDB}、GALO\cite{GALO}、Neo\cite{Neo}、
SkinnerDB\cite{SkinnerDB}等人工智能赋能的数据库新技术。

虽然研究人员利用人工智能技术针对查询优化已经进行了很多研究工作,但是它们大多是面向
关系数据库的,而尚未有研究关注针对图数据库的智能查询优化问题。因此,基于人工智能的
图数据查询优化问题亟待解决。

综上所述,由于图数据库中的RDF数据规模不断增大,需要优化SPARQL查询过程才能继续满足
应用对查询的效率要求。研究人员应用人工智能技术在SQL的查询优化中已经取得了显著的成
效,而且尚未有人应用人工智能针对SPARQL进行优化,所以本课题使用人工智能技术对SPARQL
查询进行优化是有价值有必要的。


\section{国内外研究现状}[Research status]

\subsection{人工智能在关系数据库查询的优化}[Research status in SQL]
查询优化问题已经被研究了四十多年,至今仍然是一个活跃的研究领域。该问题的组合复杂性
使得启发式算法得以普遍应用[7-9]。许多商用DBMS的查询优化策略都是基于System R [2]
或Volcano Optimizer Generator [8]中引入的思想。这些系统使用了动态规划(DP)并
结合一组规则来找到好的查询计划。这些规则缩减了潜在查询计划的搜索空间,减少了优化查询
所花费的时间,但也会降低在大型搜索空间中找到最佳查询计划的可能性。传统查询优化方法除了
搜索策略的局限性外,还存在第二个问题:它们依靠代价模型来估计执行查询的成本。这些代价模
型建立在基数估计的基础上,基数估计大多基于分位数统计,频率分析或没有理论基础的方法[16]。
基数估计中的错误通常会导致查询计划的效率不够理想。此外,传统的查询优化器无法从先前执行的
查询中学习。

受机器学习最新进展的启发,数据库研究的一个新趋势是将启发式算法用学习方法代替。因此,
Krishnan等人探索了使用学习方法合成特定数据集的连接搜索策略[10]。假设一个给定的代
价模型和计划空间,研究是否能够对一个特定数据集上所有可能的连接计划进行搜索。这一问
题的目标是从之前能够大幅度减少搜索时间的计划中学习到适合的搜索策略。Krishnan等人
的主要发现是连接排序与强化学习有深层次的算法联系,即连接排序的顺序结构也是支持强化
学习问题的结构。所以,Krishnan等人利用这一算法上的联系将强化学习嵌入到传统的查询
优化器中,建立了一个基于强化学习的优化器DQ。该优化器优化了选择-投影-连接、查询连
接以及物理操作符选择,能够根据之前执行查询计划的结果训练一个强化学习模型,从而有效
完善后续的查询操作。

Marcus等人也将深度强化学习应用于查询优化中,提出了一个自动查询优化器,使得优化器
。此外,该优化器紧密结合能够自动调整特定的数据库,无需专业数据库管理员的干预[11]
过去查询优化和执行的经验反馈,从而显著提高了查询性能。Marcus等人还提出了一个依赖
深度神经网络来生成查询执行计划的查询优化器Neo[5]。该优化器从已有的优化器生成查询
优化模型,并持续根据到来的查询学习,从成功的查询中建立模型,从失败的查询中进一步学
习调整。此外,Neo优化器能够适应不同的数据模式,且能够容忍一定的估计错误。

\subsection{SPARQL查询优化的研究}[Research status in SPARQL]
在SPARQL查询优化的方向上,也有很多研究者做出了努力。Markus在ARQ(ApacheSPARQLProcessor)
的基础上进行了改进[12],做出了optARQ原型,它通过建立选择性估计索引的方式,来最小
化查询中间结果的生成数量,从而提高查询的效率。Groppe认为SPARQL语言不够精简[13],
他提出了SPARQL语言的核心部分,并称之为coreSPARQL,它具有与SPARQL相同的表达能力,
但是消除了SPARQL的冗余语言构造,而且它的语法对机器更加友好。Groppe将SPARQL语言
转换为coreSPARQL语言,并开发了一组重写规则来优化coreSPRQL查询,以此来提高执行效率。

国内学者在SPARQL的查询优化上也做出了很多的努力。武汉大学信息资源研究中心提出了一种
优化方法[14],他们使用RDF模式信息来精简SPARQL基本图模式,然后使用B树结构快速估计
SPARQL连接图的节点大小及边权值,使用连接代价估计并结合动态规划方法找到最优逻辑查询
计划。除了这种基于传统算法的优化方法,启发式算法也得到了大量的应用,例如,有学者提出
了一种基于精英蚁群算法与权重矩阵的SPARQL查询优化算法[15],他针对不同的图形状设计了
有效的权重矩阵模型,为不同的查询形状设置了专门的优化参数,然后将权重矩阵作为蚁群算法
的输入参数,分别利用人工蚁群与精英蚁群方法对SPARQL不同形状的查询语句进行优化。

\subsection{研究现状分析}[Research status analytics]
根据国内外对SPARQL查询优化的研究内容,对研究现状的分析如下:

(1)应用人工智能技术在SQL的查询优化中已经取得了显著的成效,例如Neo查询优化器和DQ查询
优化器,其性能较传统方法有极大的提升而且还能在查询流中进行自学习,不断提升适应能力和查询
效率。

(2)现有的针对SPARQL的查询优化主要是改进代价估计模型,精简SPARQL语言和改进RDF的存储
方式,这些方法与SQL查询优化的研究一样,都存在搜索空间不足和代价估计模型不够准确的问题,而
采用人工智能技术的研究还没有。


\section{本文研究内容和结构安排}[Research content and structure]
本文的主要研究内容为基于人工智能的SPARQL的查询优化技术,采用了强化学习对SPARQL中的连
接顺序进行优化,文章的主要章节如下:

第一章为绪论部分,开始介绍了本课题的研究背景和本研究的目的意义,然后对目前国内外对SPARQL
查询优化的现状进行了分析和总结,最后给出了本文的研究内容和结构安排。

第二章介绍了本文用到的查询优化的基础知识。......

第三章介绍了基于强化学习的SPARQL查询优化的方法。......

第四章介绍了对方法的实现,测试和结果分析。......

\chapter{预备知识}[Preliminary knowledge]

\section{RDF图模型}

RDF全称为资源描述框架(resource description framework),是万维网联盟(World Wide
 Web consortium,简称W3C)制定的在语义Web(semantic Web)上表示和交换机器可理解
 (machine-understandable)信息的标准数据模型[23]。在互联网中,所有能够用RDF数据
 表示的对象都称之为资源,例如所有的事物、概念和信息等。资源通常情况下使用唯一资源标识
 符(URI)来表示。在RDF图中,每个资源具有一个HTTP URI作为其唯一id;图...

 \begin{definition}[RDF项(RDF Term)]    
    用I来表示IRI的集合
    用RDF-L来表示文本信息(Literals)的集合
    用RDF-B来表示空值(Blank nodes)的集合
    用RDF-T来表示RDF项(RDF Term)
\end{definition}

\begin{definition}[(RDF三元组)]    
    RDF三元组是形如(s,p,o)的数据组合,又被称为声明。其中s代表主体(subject),p代表属性(property),
    o代表客体(object)。三元组的选取区间可以表示为$(I\bigcup RDF-B)X(I\bigcup RDF-B)X(I\bigcup RDF-B\bigcup RDF-L)$。
    
    例如,对于三元组<Tony, graduateFrom, HIT>,Tony代表主体,graduateFrom(毕业于)代表属性,HIT代表客体。三元组的意思是
    Tony毕业于HIT。
\end{definition}

\section{SPARQL语言与查询过程}

SPARQL(simple protocol and RDF query language)是W3C提出的用于检索RDF数据的形式化查询语言,
也是目前使用最广泛的查询语言。类似于SQL,SPARQL支持多种查询格式,例如SELECT、ASK、DESCRIBE、CONSTRUCT。
其中,SELECT查询格式用于标准查询,以标准的 SPARQL XML 结果格式返回查询结果。ASK 查询返回结果是 yes 或 no,没有具体内
容。DESCRIBE 用于提取本体和实例数据的一部分,返回一个图形,其中包含和图形模式匹配的节点的相关信息。
\begin{definition}[(三元组模式)]    
    一个三元组模式(Triple Pattern)是集合$(RDF-T \bigcup V)X(I \bigcup V)X(RDF-T \bigcup V)$的一项,其中V代表变量。
\end{definition}
\begin{definition}[基本图模式(BGP)]    
    一个三元组模式(Triple Pattern)是三元组模式(Triple Pattern)集合的一项。
\end{definition}
图...是一个示例SPARQL查询,其中WHERE子句是查询的主要组成部分,包含6个基本的三元组模式(triple pattern),其中三元组模式通过
对RDF三元组中的主体,属性或客体进行变量替换得到。

SPARQL查询过程 中,用户(包括人和机器)通过一系列接口与系统进行交互,接口将查询请求送入
SPARQL查询处理器,调用底层的RDF存储获取相关的结果记录。SPARQL通过查询器扫描关键词,并且根据
标准解析查询序列验证RDF三元组的有效性。如果查询不正确,则在处理的过程中及时通过接口为用户返
回错误信息,如图\ref{SPARQL查询过程}所示:

在 SPARQL 处理器中,首先利用解析器对查询语句进行解析,判断是否存在语法错误。接着在重写查询
阶段,以规则为基础重新优化查询语句。最后通过执行 QEP(查询执行计划)发生器产生的计划获取 RDF 数
据并通过接口返回给用户。
\begin{figure}[h]
\centering
\includegraphics[width = 1\textwidth]{SPARQL}
\caption{SPARQL查询过程}
\label{SPARQL查询过程}
\end{figure}

\section{强化学习}[Reinforcement Learning]
\subsection{概述}[introduction]
强化学习(Reinforcement learning)\cite{RLIntroduction}是机器学习中的一个领域,这种方法主要思想是通过与环境的交互来学习。其来源于心理学中的行为主义理论,即有机体如何在环境给予的奖励或惩罚的刺激下,逐步形成对刺激的预期,产生能获得最大利益的习惯性行为。例如当婴儿开始学习走路时,他刚开始并不能掌握平衡,并不会直接学会走路,但他通过与周围环境的交互获得反馈,例如摔倒会给他带来疼痛感(惩罚),顺利走出第一步保持了平衡(奖励),婴儿就会知道每个动作的后果(获得惩罚或奖励),从而逐渐选择奖励最多的动作,直到完全学会走路(奖励最大化)。当我们从计算机科学的角度看待这种方法时,我们可以将其解释为一个函数。该函数尝试使奖励信号最大化,而无需告知必须采取何种行动。

很多形式的有监督机器学习方法通常都有很多的输入输出对,每一个输入输出都是提前备好的训练数据集,训练过程会被告知要采取哪些动作,但是强化学习不同于前者的是它必须通过尝试不同的动作,根据与环境的交互获得动作的反馈,从而确定采取哪些动作能产生最大的回报。强化学习更加专注于在线规划,需要在探索(在未知的领域)和遵从(现有知识)之间找到平衡。

\subsection{马尔科夫决策过程}[MDP]
从技术角度来看,强化学习是一种随机优化方法,它可以表述为马尔可夫决策过程(MDP)\cite{MDP}。其基本思想是,代理(agent)采取一系列行动,以优化MDP模型中的给定目标。

\section{图片}[Pictures]
图应有自明性。插图应与文字紧密配合,文图相符,内容正确。选图要力求精练,插图、照
片应完整清晰。机械工程图:采用第一角投影法,严格按照GB4457~GB131-83《机械制图》
标准规定。数据流程图、程序流程图、系统流程图等按GB1526-89标准规定。电气图:图形
符号、文字符号等应符合附录3所列有关标准的规定。流程图:必须采用结构化程序并正确
运用流程框图。对无规定符号的图形应采用该行业的常用画法。坐标图的坐标线均用细实线
,粗细不得超过图中曲线;有数字标注的坐标图,必须注明坐标单位。照片图要求主题和主
要显示部分的轮廓鲜明,便于制版。如用放大或缩小的复制品,必须清晰,反差适中。照片
上应有表示目的物尺寸的标度。引用文献中的图时,除在正文文字中标注参考文献序号以外
,还必须在中、英文表题的右上角标注参考文献序号。

\subsection{博士毕业论文双语题注}[Doctoral picture example]
\begin{figure}[htpb]
\centering
\includegraphics[width = 0.4\textwidth]{golfer}
\bicaption[golfer1]{}{打高尔夫球球的人(博士论文双语题注)}{Fig.$\!$}{The person playing golf (Doctoral thesis)}
\end{figure}

每个图均应有图题(由图序和图名组成),图题不宜有标点符号,图名在图序之后空1个半
角字符排写。图序按章编排,如第1章第一个插图的图号为“图1-1”。图题置于图下,硕士论
文只用中文,博士论文用中、英两种文字,居中书写,中文在上,要求中文用宋体5号字,
英文用Times New Roman 5号字。有图注或其它说明时应置于图题之上。引用图应注明出处
,在图题右上角加引用文献号。图中若有分图时,分图题置于分图之下或图题之下,可以只
用中文书写,分图号用a)、b)等表示。图中各部分说明应采用中文(引用的外文图除外)或
数字符号,各项文字说明置于图题之上(有分图时,置于分图题之上)。图中文字用宋体、
Times New Roman字体,字号尽量采用5号字(当字数较多时可用小5号字,以清晰表达为原
则,但在一个插图内字号要统一)。同一图内使用文字应统一。图表中物理量、符号用斜体
。
\subsection{本硕论文题注}[Other picture example]
\begin{figure}[h]
\centering
\includegraphics[width = 0.4\textwidth]{golfer}
\caption{打高尔夫球的人,硕士论文要求只用汉语}
\end{figure}

\subsection{并排图和子图}[Abreast-picture and Sub-picture example]
\subsubsection{并排图}[Abreast-picture example]

使用并排图时,需要注意对齐方式。默认情况是中部对齐。这里给出中部对齐、顶部对齐
、图片底部对齐三种常见方式。其中,底部对齐方式有一个很巧妙的方式,将长度比较小
的图放在左面即可。

\begin{figure}[htbp]
\centering
\begin{minipage}{0.4\textwidth}
\centering
\includegraphics[width=\textwidth]{golfer}
\bicaption[golfer2]{}{打高尔夫球的人}{Fig.$\!$}{The person playing golf}
\end{minipage}
\centering
\begin{minipage}{0.4\textwidth}
\centering
\includegraphics[width=\textwidth]{golfer}
\bicaption[golfer3]{}{打高尔夫球的人。注意,这里默认居中}{Fig.$\!$}{The person playing golf. Please note that, it is vertically center aligned by default.}
\end{minipage}
\end{figure}

\begin{figure}[htbp]
\centering
\begin{minipage}[t]{0.4\textwidth}
\centering
\includegraphics[width=\textwidth]{golfer}
\bicaption[golfer5]{}{打高尔夫球的人}{Fig.$\!$}{The person playing golf}
\end{minipage}
\centering
\begin{minipage}[t]{0.4\textwidth}
\centering
\includegraphics[width=\textwidth]{golfer}
\bicaption[golfer8]{}{打高尔夫球的人。注意,此图是顶部对齐}{Fig.$\!$}{The person playing golf. Please note that, it is vertically top aligned.}
\end{minipage}
\end{figure}

\begin{figure}[htbp]
\centering
\begin{minipage}[t]{0.4\textwidth}
\centering
\includegraphics[width=\textwidth,height=\textwidth]{golfer}
\bicaption[golfer9]{}{打高尔夫球的人。注意,此图对齐方式是图片底部对齐}{Fig.$\!$}{The person playing golf. Please note that, it is vertically bottom aligned for figure.}
\end{minipage}
\centering
\begin{minipage}[t]{0.4\textwidth}
\centering
\includegraphics[width=\textwidth]{golfer}
\bicaption[golfer6]{}{打高尔夫球的人}{Fig.$\!$}{The person playing golf}
\end{minipage}
\end{figure}

\subsubsection{子图}[Sub-picture example]
注意:子图题注也可以只用中文。规范规定“分图题置于分图之下或图题之下”,但没有给出具体的格式要求。
没有要求的另外一个说法就是“无论什么格式都不对”。
所以只有在一个图中有标注“a),b)”,无法使用\cs{subfigure}的情况下,使用最后一个图例中的格式设置方法,否则不要使用。
为了应对“无论什么格式都不对”,这个子图图题使用“minipage”和“description”环境,宽度,对齐方式可以按照个人喜好自由设置,是否使用双语子图图题也可以自由设置。

\begin{figure}[!h]
\setlength{\subfigcapskip}{-1bp}
\centering
\begin{minipage}{\textwidth}
\centering
\subfigure{\label{golfer41}}\addtocounter{subfigure}{-2}
\subfigure[The person playing golf]{\subfigure[打高尔夫球的人~1]{\includegraphics[width=0.4\textwidth]{golfer}}}
\hspace{2em}
\subfigure{\label{golfer42}}\addtocounter{subfigure}{-2}
\subfigure[The person playing golf]{\subfigure[打高尔夫球的人~2]{\includegraphics[width=0.4\textwidth]{golfer}}}
\end{minipage}
\centering
\begin{minipage}{\textwidth}
\centering
\subfigure{\label{golfer43}}\addtocounter{subfigure}{-2}
\subfigure[The person playing golf]{\subfigure[打高尔夫球的人~3]{\includegraphics[width=0.4\textwidth]{golfer}}}
\hspace{2em}
\subfigure{\label{golfer44}}\addtocounter{subfigure}{-2}
\subfigure[The person playing golf. Here, 'hang indent' and 'center last line' are not stipulated in the regulation.]{\subfigure[打高尔夫球的人~4。注意,规范中没有明确规定要悬挂缩进、最后一行居中。]{\includegraphics[width=0.4\textwidth]{golfer}}}
\end{minipage}
\vspace{0.2em}
\bicaption[golfer4]{}{打高尔夫球的人}{Fig.$\!$}{The person playing gol}
\end{figure}

\begin{figure}[t]
  \centering
  \begin{minipage}{.7\linewidth}
    \setlength{\subfigcapskip}{-1bp}
    \centering
    \begin{minipage}{\textwidth}
      \centering
      \subfigure{\label{golfer45}}\addtocounter{subfigure}{-2}
      \subfigure[The person playing golf]{\subfigure[打高尔夫球的人~1]{\includegraphics[width=0.4\textwidth]{golfer}}}
      \hspace{4em}
      \subfigure{\label{golfer46}}\addtocounter{subfigure}{-2}
      \subfigure[The person playing golf]{\subfigure[打高尔夫球的人~2]{\includegraphics[width=0.4\textwidth]{golfer}}}
    \end{minipage}
    \vskip 0.2em
  \wuhao 注意:这里是中文图注添加位置(我工要求,图注在图题之上)。
    \vspace{0.2em}
\bicaption[golfer47]{}{打高尔夫球的人。注意,此处我工有另外一处要求,子图图题可以位于主图题之下。但由于没有明确说明位于下方具体是什么格式,所以这里不给出举例。}{Fig.$\!$}{The person playing golf. Please note that, although it is appropriate to put subfigures' captions under this caption as stipulated in regulation, but its format is not clearly stated.}
  \end{minipage}
\end{figure}

\begin{figure}[t]
\centering
\begin{tikzpicture}
	\node[anchor=south west,inner sep=0] (image) at (0,0) {\includegraphics[width=0.3\textwidth]{golfer}};
	\begin{scope}[x={(image.south east)},y={(image.north west)}]
		\node at (0.3,0.5) {a)};
		\node at (0.8,0.2) {b)};
	\end{scope}
\end{tikzpicture}
\bicaption[golfer0]{}{打高尔夫球球的人(博士论文双语题注)}{Fig.$\!$}{The person playing golf (Doctoral thesis)}
\vskip -0.4em
 \hspace{2em}
\begin{minipage}[t]{0.3\textwidth}
\wuhao \setlist[description]{font=\normalfont}
	\begin{description}
		\item[a)]子图图题
	\end{description}
 \end{minipage}
 \hspace{2em}
 \begin{minipage}[t]{0.3\textwidth}
\wuhao \setlist[description]{font=\normalfont}
	\begin{description}
		\item[b)]子图图题
		\item[b)]Subfigure caption
	\end{description}
\end{minipage}
\end{figure}


\begin{figure}[!h]
	\centering
	\begin{sideways}
		\begin{minipage}{\textheight}
			\centering
			\fbox{\includegraphics[width=0.2\textwidth]{golfer}}
			\fbox{\includegraphics[width=0.2\textwidth]{golfer}}
			\fbox{\includegraphics[width=0.2\textwidth]{golfer}}
			\fbox{\includegraphics[width=0.2\textwidth]{golfer}}
			\fbox{\includegraphics[width=0.2\textwidth]{golfer}}
			\fbox{\includegraphics[width=0.2\textwidth]{golfer}}
			\fbox{\includegraphics[width=0.2\textwidth]{golfer}}
\bicaption[golfer7]{}{打高尔夫球的人(非规范要求)}{Fig.$\!$}{The person playing golf (Not stated in the regulation)}
		\end{minipage}
	\end{sideways}
\end{figure}

\clearpage

如果不想让图片浮动到下一章节,那么在此处使用\cs{clearpage}命令。

\section{如何做出符合规范的漂亮的图}
关于作图工具在后文\ref{drawtool}中给出一些作图工具的介绍,此处不多言。
此处以R语言和Tikz为例说明如何做出符合规范的图。

\subsection{Tikz作图举例}
使用Tikz作图核心思想是把格式、主题、样式与内容分离,定义在全局中。
注意字体设置可以有两种选择,如何字少,用五号字,字多用小五。
使用Tikz作图不会出现字体问题,字体会自动与正文一致。

\begin{figure}[thb!]
  \centering
      \begin{tikzpicture}[xscale=0.8,yscale=0.3,rotate=90]
        \small
	\draw (-22,6.5) node[refcell]{参考基因组};
	\draw[refline] (-23, 5) -- (27, 5);
	\draw (-22,3.75) node[tscell]{肿瘤样本};
	\draw (-20,3.75) node[tncell]{正常细胞};
	\draw[tnline] (-21, 2.5) -- (27, 2.5);
	\draw (-20,1.25) node[ttcell]{肿瘤细胞};
	\rcell{2}{6};
	\draw[fakeevolve] (4.5, 5.25) -- (4.5, 4.8);
	\ncell{2}{4};
	\draw[evolve] (4.5, 3) .. controls (4.5,2.8) and (-3.5,2.9) ..  (-3.5, 2);
	\draw[evolve] (4.5, 3) .. controls (4.5,2.8) and (11.5,2.9) .. (11.5, 2);
	\tcellone{-6}{1.5};
	\draw (-9, 2) node[ttcell]{1};
	\draw[evolve] (-3.5, 0) .. controls (-3.5,-0.2) and (-12,-0.1) .. (-12, -1.5);
	\draw[evolve] (-3.5, 0) .. controls (-3.5,-0.2) and (1.5,-0.1) .. (1.5, -1.5);
	\tcellthree{7}{1.5};
	\draw (4, 2) node[ttcell]{2};
	\draw[evolve] (11, 0.5) .. controls (11,0.3) and (19,0.4) .. (19, -1.5);
	\tcellfive{-16}{-2};
	\draw (-19, -1.5) node[ttcell]{3};
	\tcelltwo{-1}{-2};
	\draw (-4, -1.5) node[ttcell]{4};
	\tcellfour{12}{-2};
	\draw (9, -1.5) node[ttcell]{5};
      \end{tikzpicture}
  \begin{minipage}{.9\linewidth}
      \vskip 0.2em
      \wuhao 图中,带有箭头的淡蓝色箭头表示肿瘤子种群的进化方向。一般地,从肿瘤组织中取用于进行二代测序的样本中含有一定程度的正常细胞污染,因此肿瘤的样本中含有正常细胞和肿瘤细胞。每一个子种群的基因组的模拟过程是把生殖细胞变异和体细胞变异加入到参考基因组中。
      \vspace{0.6em}
  \end{minipage}
\bicaption[tumor]{}{肿瘤组织中各个子种群的进化示意图}{Fig.$\!$}{The diagram of tumor subpopulation evolution process}
\end{figure}

\subsection{R作图}
R是一种极具有代表性的典型的作图工具,应用广泛。
与Tikz图~\ref{tumor}~不同,R作图分两种情况:(1)可以转换为Tikz码;(2)不可转换为Tikz码。
第一种情况图形简单,图形中不含有很多数据点,使用R语言中的Tikz包即可。
第二种情况是图形复杂,含有海量数据点,这时候不要转成Tikz矢量图,这会使得论文体积巨大。
推荐使用pdf或png非矢量图形。
使用非矢量图形时要注意选择好字号(五号或小五),和字体(宋体、新罗马)然后选择生成图形大小,注意此时在正文中使用\cs{includegraphics}命令导入时,不要像导入矢量图那样控制图形大小,使用图形的原本的
宽度和高度,这样就确保了非矢量图形中的文字与正文一致了。

为了控制\hithesis\ 的大小,此处不给出具体举例,

\section{表格}

表应有自明性。表格不加左、右边线。表的编排建议采用国际通行的三线表。表中文字用宋
体~5~号字。每个表格均应有表题(由表序和表名组成)。表序一般按章编排,如第~1~章第
一个插表的序号为“表~1-1”等。表序与表名之间空一格,表名中不允许使用标点符号,表名
后不加标点。表题置于表上,硕士学位论文只用中文,博士学位论文用中、英文两种文字居
中排写,中文在上,要求中文用宋体~5~号字,英文用新罗马字体~5~号字。表头设计应简单
明了,尽量不用斜线。表头中可采用化学符号或物理量符号。


\subsection{普通表格的绘制方法}[Methods of drawing normal tables]

表格应具有三线表格式,因此需要调用~booktabs~宏包,其标准格式如表~\ref{table1}~所示。
\begin{table}[htbp]
\bicaption[table1]{}{符合研究生院绘图规范的表格}{Table$\!$}{Table in agreement of the standard from graduate school}
\vspace{0.5em}\centering\wuhao
\begin{tabular}{ccccc}
\toprule[1.5pt]
$D$(in) & $P_u$(lbs) & $u_u$(in) & $\beta$ & $G_f$(psi.in)\\
\midrule[1pt]
 5 & 269.8 & 0.000674 & 1.79 & 0.04089\\
10 & 421.0 & 0.001035 & 3.59 & 0.04089\\
20 & 640.2 & 0.001565 & 7.18 & 0.04089\\
\bottomrule[1.5pt]
\end{tabular}
\end{table}
全表如用同一单位,则将单位符号移至表头右上角,加圆括号。表中数据应准确无误,书写
清楚。数字空缺的格内加横线“-”(占~2~个数字宽度)。表内文字或数字上、下或左、右
相同时,采用通栏处理方式,不允许用“〃”、“同上”之类的写法。表内文字说明,起行空一
格、转行顶格、句末不加标点。如某个表需要转页接排,在随后的各页上应重复表的编号。
编号后加“(续表)”,表题可省略。续表应重复表头。

\subsection{长表格的绘制方法}[Methods of drawing long tables]

长表格是当表格在当前页排不下而需要转页接排的情况下所采用的一种表格环境。若长表格
仍按照普通表格的绘制方法来获得,其所使用的\verb|table|浮动环境无法实现表格的换页
接排功能,表格下方过长部分会排在表格第1页的页脚以下。为了能够实现长表格的转页接
排功能,需要调用~longtable~宏包,由于长表格是跨页的文本内容,因此只需要单独的
\verb|longtable|环境,所绘制的长表格的格式如表~\ref{table2}~所示。

注意,长表格双语标题的格式。

\vspace{-1.5bp}
\ltfontsize{\wuhao[1.667]}
\wuhao[1.667]\begin{longtable}{ccc}%
\longbionenumcaption{}{{\wuhao 中国省级行政单位一览
}\label{table3}}{Table$\!$}{}{{\wuhao Overview of the provincial administrative
unit of China}}{-0.5em}{3.15bp}\\
%\caption{\wuhao 中国省级行政单位一览}\\
\toprule[1.5pt] 名称 & 简称 & 省会或首府  \\ \midrule[1pt]
\endfirsthead
\multicolumn{3}{r}{表~\thetable(续表)}\vspace{0.5em}\\
\toprule[1.5pt] 名称 & 简称 & 省会或首府  \\ \midrule[1pt]
\endhead
\bottomrule[1.5pt]
\endfoot
北京市 & 京 & 北京\\
天津市 & 津 & 天津\\
河北省 & 冀 & 石家庄市\\
山西省 & 晋 & 太原市\\
内蒙古自治区 & 蒙 & 呼和浩特市\\
辽宁省 & 辽 & 沈阳市\\
吉林省 & 吉 & 长春市\\
黑龙江省 & 黑 & 哈尔滨市\\
上海市 & 沪/申 & 上海\\
江苏省 & 苏 & 南京市\\
浙江省 & 浙 & 杭州市\\
安徽省 & 皖 & 合肥市\\
福建省 & 闽 & 福州市\\
江西省 & 赣 & 南昌市\\
山东省 & 鲁 & 济南市\\
河南省 & 豫 & 郑州市\\
湖北省 & 鄂 & 武汉市\\
湖南省 & 湘 & 长沙市\\
广东省 & 粤 & 广州市\\
广西壮族自治区 & 桂 & 南宁市\\
海南省 & 琼 & 海口市\\
重庆市 & 渝 & 重庆\\
四川省 & 川/蜀 & 成都市\\
贵州省 & 黔/贵 & 贵阳市\\
云南省 & 云/滇 & 昆明市\\
西藏自治区 & 藏 & 拉萨市\\
陕西省 & 陕/秦 & 西安市\\
甘肃省 & 甘/陇 & 兰州市\\
青海省 & 青 & 西宁市\\
宁夏回族自治区 & 宁 & 银川市\\
新疆维吾尔自治区 & 新 & 乌鲁木齐市\\
香港特别行政区 & 港 & 香港\\
澳门特别行政区 & 澳 & 澳门\\
台湾省 & 台 & 台北市\\
\end{longtable}\normalsize
\vspace{-1em}

此长表格~\ref{table2}~第~2~页的标题“编号(续表)”和表头是通过代码自动添加上去的,无需人工添加,若表格在页面中的竖直位置发生了变化,长表格在第~2~页
及之后各页的标题和表头位置能够始终处于各页的最顶部,也无需人工调整,\LaTeX~系统的这一优点是~word~等软件所无法比拟的。

\subsection{列宽可调表格的绘制方法}[Methods of drawing tables with adjustable-width columns]
论文中能用到列宽可调表格的情况共有两种,一种是当插入的表格某一单元格内容过长以至
于一行放不下的情况,另一种是当对公式中首次出现的物理量符号进行注释的情况,这两种
情况都需要调用~tabularx~宏包。下面将分别对这两种情况下可调表格的绘制方法进行阐述
。
\subsubsection{表格内某单元格内容过长的情况}[The condition when the contents in
some cells of tables are too long]
首先给出这种情况下的一个例子如表~\ref{table3}~所示。
\begin{table}[htbp]
  \centering
\bicaption[table4]{}{最小的三个正整数的英文表示法}{Table$\!$}{The English construction of the smallest three positive integral numbers}\vspace{0.5em}\wuhao
\begin{tabularx}{0.7\textwidth}{llX}
\toprule[1.5pt]
Value & Name & Alternate names, and names for sets of the given size\\\midrule[1pt]
1 & One & ace, single, singleton, unary, unit, unity\\
2 & Two & binary, brace, couple, couplet, distich, deuce, double, doubleton, duad, duality, duet, duo, dyad, pair, snake eyes, span, twain, twosome, yoke\\
3 & Three & deuce-ace, leash, set, tercet, ternary, ternion, terzetto, threesome, tierce, trey, triad, trine, trinity, trio, triplet, troika, hat-trick\\\bottomrule[1.5pt]
\end{tabularx}
\end{table}
tabularx环境共有两个必选参数:第1个参数用来确定表格的总宽度,第2个参数用来确定每
列格式,其中标为X的项表示该列的宽度可调,其宽度值由表格总宽度确定。标为X的列一般
选为单元格内容过长而无法置于一行的列,这样使得该列内容能够根据表格总宽度自动分行
。若列格式中存在不止一个X项,则这些标为X的列的列宽相同,因此,一般不将内容较短的
列设为X。标为X的列均为左对齐,因此其余列一般选为l(左对齐),这样可使得表格美观
,但也可以选为c或r。

\subsubsection{对物理量符号进行注释的情况}[The condition when physical symbols
need to be annotated]

为使得对公式中物理量符号注释的转行与破折号“———”后第一个字对齐,此处最好采用表格
环境。此表格无任何线条,左对齐,且在破折号处对齐,一共有“式中”二字、物理量符号和
注释三列,表格的总宽度可选为文本宽度,因此应该采用\verb|tabularx|环境。由
\verb|tabularx|环境生成的对公式中物理量符号进行注释的公式如式(\ref{eq:1})所示。
\begin{equation}\label{eq:1}
\ddot{\boldsymbol{\rho}}-\frac{\mu}{R_{t}^{3}}\left(3\mathbf{R_{t}}\frac{\mathbf{R_{t}\rho}}{R_{t}^{2}}-\boldsymbol{\rho}\right)=\mathbf{a}
\end{equation}
\begin{tabularx}{\textwidth}{@{}l@{\quad}r@{———}X@{}}
式中& $\boldsymbol{\rho}$ &追踪飞行器与目标飞行器之间的相对位置矢量;\\
&  $\boldsymbol{\ddot{\rho}}$&追踪飞行器与目标飞行器之间的相对加速度;\\
&  $\mathbf{a}$   &推力所产生的加速度;\\
&  $\mathbf{R_t}$ & 目标飞行器在惯性坐标系中的位置矢量;\\
&  $\omega_{t}$ & 目标飞行器的轨道角速度;\\
&  $\mathbf{g}$ & 重力加速度,$=\frac{\mu}{R_{t}^{3}}\left(
3\mathbf{R_{t}}\frac{\mathbf{R_{t}\rho}}{R_{t}^{2}}-\boldsymbol{\rho}\right)=\omega_{t}^{2}\frac{R_{t}}{p}\left(
3\mathbf{R_{t}}\frac{\mathbf{R_{t}\rho}}{R_{t}^{2}}-\boldsymbol{\rho}\right)$,这里~$p$~是目标飞行器的轨道半通径。
\end{tabularx}\vspace{3.15bp}
由此方法生成的注释内容应紧邻待注释公式并置于其下方,因此不能将代码放入
\verb|table|浮动环境中。但此方法不能实现自动转页接排,可能会在当前页剩余空间不够
时,全部移动到下一页而导致当前页出现很大空白。因此在需要转页处理时,还请您手动将
需要转页的代码放入一个新的\verb|tabularx|环境中,将原来的一个\verb|tabularx|环境
拆分为两个\verb|tabularx|环境。

\subsubsection{排版横版表格的举例}[An example of landscape table]

\begin{table}[p]
\centering
\begin{sideways}
\begin{minipage}{\textheight}
\bicaption[table2]{}{不在规范中规定的横版表格}{Table$\!$}{A table style which is not stated in the regulation}
\vspace{0.5em}\centering\wuhao
\begin{tabular}{ccccc}
\toprule[1.5pt]
$D$(in) & $P_u$(lbs) & $u_u$(in) & $\beta$ & $G_f$(psi.in)\\
\midrule[1pt]
 5 & 269.8 & 0.000674 & 1.79 & 0.04089\\
10 & 421.0 & 0.001035 & 3.59 & 0.04089\\
20 & 640.2 & 0.001565 & 7.18 & 0.04089\\
\bottomrule[1.5pt]
\end{tabular}
\end{minipage}
\end{sideways}
\end{table}


\section{公式}
与正常\LaTeX\ 使用方法一致,此处略。关于公式中符号样式的定义在`hithesis.sty'有示
例。

\section{其他杂项}[Miscellaneous]

\subsection{右翻页}[Open right]

对于双面打印的论文,强制使每章的标题页出现右手边为右翻页。
规范中没有明确规定是否是右翻页打印。
模板给出了右翻页选项。
为了应对用户的个人喜好,在希望设置成右翻页的位置之前添加\cs{cleardoublepage}命令即可。

\subsection{算法}[Algorithms]
我工算法有以下几大特点。

(1)算法不在规范中要求。

(2)算法常常被使用(至少计算机学院)。

(3)格式乱,甚至出现了每个实验室的格式要求都不一样。

此处不给出示例,因为没法给,在
\href{https://github.com/dustincys/PlutoThesis}{https://github.com/dustincys/PlutoThesis}
的readme文件中有不同实验室算法要求说明。

\subsection{脚注}[Footnotes]
不在再规范\footnote{规范是指\PGR\ 和\UGR}中要求,模板默认使用清华大学的格式。

\subsection{源码}[Source code]
也不在再规范中要求。如果有需要最好使用minted包,但在编译的时候需要添加“
-shell-escape”选项且安装pygmentize软件,这些不在模板中默认载入,如果需要自行载入
。
\subsection{思源宋体}[Siyuan font]
如果要使用思源字体,需要思源字体的定义文件,此文件请到模板的开发版网址github:
\href{https://gihitb.com/dustincys/hithesis}{https://gihitb.com/dustincys/hithesis}
或者oschia:
\href{https://git.oschina.net/dustincys/hithesis}{https://git.oschina.net/dustincys/hithesis}
处下载。

\subsection{专业绘图工具}[Processional drawing tool]
\label{drawtool}
推荐使用tikz包,使用tikz源码绘图的好处是,图片中的字体与正文中的字体一致。具体如
何使用tikz绘图不属于模板范畴。
tikz适合用来画不需要大量实验数据支撑示意图。但R语言等专业绘图工具具有画出各种、
专业、复杂的数据图。R语言中有tikz包,能自动生成tikz码,这样tikz几乎无所不能。
对于排版有极致追求的小伙伴,可以参考
\href{http://www.texample.net/tikz/resources/}{http://www.texample.net/tikz/resources/}
中所列工具,几乎所有作图软件所作的图形都可转成tikz,然后可以自由的在tikz中修改
图中内容,定义字体等等。实现前文窝工规范中要求的图中字体的一致性的终极目标。


\subsection{术语词汇管理}[Manage glossaries]
推荐使用glossaries包管理术语、缩略语,可以自动生成首次全写,非首次缩写。

\subsection{\TeX\ 源码编辑器}[\TeX editor]
推荐:(1)付费软件Winedt;(2)免费软件kile;(3)vim或emaces或sublime等神级编
译器(需要配置)。

\subsection{\LaTeX\ 排版重要原则}[\LaTeX\ typesetting rules]
格式和内容分离是\LaTeX\ 最大优势,所有多次出现的内容、样式等等都可以定义为简单命
令、环境。这样的好处是方便修改、管理。例如,如果想要把所有的表示向量的符号由粗体
\cs{mathbf}变换到花体\cs{mathcal},只需修改该格式的命令的定义部分,不需要像MS
word那样处处修改。总而言之,使用自定义命令和环境才是正确的使用\LaTeX\ 的方式。

\section{关于捐助}
各位刀客和大侠如用的嗨,要解囊相助,请微信或支付宝参照图
~\ref{wct5}~到图~\ref{zfb}~中提示操作(二维码被矢量化后之后去
除了头像等冗余无用的部分~)。

\begin{figure}[!h]
\setlength{\subfigcapskip}{-1bp}
\centering
\subfigure{\label{wct5}}\addtocounter{subfigure}{-1}
\subfigure[如果用的嗨,微信扫码捐助5元]{\includegraphics[width=0.4\textwidth]{wct5}}
\hspace{2em}
\subfigure{\label{wct10}}\addtocounter{subfigure}{-1}
\subfigure[如果用的非常嗨,微信扫码捐助10元]{\includegraphics[width=0.4\textwidth]{wct10}}
\subfigure{\label{wct1}}\addtocounter{subfigure}{-1}
\subfigure[那个,看在熬夜写代码的份上,微信扫码捐助1元吧]{\includegraphics[width=0.4\textwidth]{wct1}}
\hspace{2em}
\subfigure{\label{zfb}}\addtocounter{subfigure}{-1}
\subfigure[或者支付宝不限额度]{\includegraphics[width=0.4\textwidth]{zfb}}
\vspace{0.2em}
\bicaption[Donation]{}{捐助,注意此处是子图只用汉语图题的形式,我工规定可以不用
英语图题}{Fig.$\!$}{Donation, please note that it is OK to use Chinese caption
only}
\end{figure}


% Local Variables:
% TeX-master: "../main"
% TeX-engine: xetex
% End:
